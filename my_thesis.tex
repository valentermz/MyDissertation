\documentclass[phd,tocprelim]{cornell}
 \let\ifpdf\relax % to avoid name clash error
%
% tocprelim option must be included to put the roman numeral pages in the
% table of contents
%
% The cornellheadings option will make headings completely consistent with
% guidelines.
%
% This sample document was originally provided by Blake Jacquot, and
% fixed up by Andrew Myers.
%
% Some possible packages to include
% \usepackage{graphicx,pstricks}
% \usepackage{graphics}
% \usepackage{moreverb}
% \usepackage{subfigure}
% \usepackage{epsfig}
% \usepackage{subfigure}
% \usepackage{hangcaption}
% \usepackage{txfonts}
% \usepackage{palatino}

\usepackage{valente}
 \hypersetup{pdftitle={My Thesis},pdfauthor={Valente Ramirez}}

%if you're having problems with overfull boxes, you may need to increase
%the tolerance to 9999
%\tolerance=9999

\bibliographystyle{alpha} % What is required by the Graduate School?
% \bibliographystyle{IEEEbib}

\renewcommand{\caption}[1]{\singlespacing\hangcaption{#1}\normalspacing}
\renewcommand{\topfraction}{0.85}
\renewcommand{\textfraction}{0.1}
\renewcommand{\floatpagefraction}{0.75}

% Personal information
\title {Quadratic vector fields on \texorpdfstring{$\C^2$}{C2}:\\rigidity and their analytic invariants}
	% Warning: If the title is modified, make sure it matches the \@abstracttitle in cornell.cls
\author {Valente Ram\'{i}rez Garc\'{i}a Luna}
\conferraldate {August}{2017}
\degreefield {Ph.D.}
\copyrightholder{Valente Ram\'{i}rez Garc\'{i}a Luna}
\copyrightyear{2017}

\begin{document}

\maketitle
% \makecopyright
% 
% \begin{abstract}
% Your abstract goes here. Make sure it sits inside the brackets. If not,
% your biosketch page may not be roman numeral iii, as required by the
% graduate school.
% \end{abstract}
% 
% \begin{biosketch}
% Your biosketch goes here. Make sure it sits inside
% the brackets.
% \end{biosketch}
% 
% \begin{dedication}
% Dedication
% \end{dedication}
% 
% \begin{acknowledgements}
% Your acknowledgements go here. Make sure it sits inside the brackets.
% \end{acknowledgements}

\contentspage
% \tablelistpage
% \figurelistpage

\normalspacing \setcounter{page}{1} \pagenumbering{arabic}
\pagestyle{cornell} \addtolength{\parskip}{0.5\baselineskip}


\addcontentsline{toc}{chapter}{Preface}
\chapter*{Preface}

The results presented in this thesis belong to the theory of \textit{holomorphic foliations}. More precisely, to the study of polynomial vector fields on $\C^2$ and the holomorphic foliations that such vector fields define on $\CP^2$. The theory of holomorphic foliations is a rich area of mathematics where several other branches of mathematics come together: ordinary differential equations, one-dimensional complex dynamics, complex analytic and complex algebraic geometry, singularity theory, topology. 

The origins of the theory of holomorphic foliations trace back to the beginning of the \textit{qualitative theory} of ordinary differential equations, pioneered by Poincar\'{e} towards the end of the XIX$^{\text{th}}$ century. 
%
%cit: H. Poincare, "Mémoire sur les courbes définiés par une équation différentielle" J. de Math--a series of papers between 1881 and 1886. cf. https://www.encyclopediaofmath.org/index.php/Qualitative_theory_of_differential_equations
%
The brilliant insight of Poincar\'{e} was that differential equations belong not only to the realm of analysis, but also to geometry. For example, providing a non-autonomous differential equation of the form $\displaystyle\frac{dx}{dt}=f(x,t)$ is equivalent to giving a vector field, a much more geometric object, of the form
 \[ v = f(x,t)\frac{\partial}{\partial x} + \frac{\partial}{\partial t}, \qquad (x,t)\in U\subset\R^2, \]
in some appropriate open set $U$. A solution to the given differential equation corresponds to a smooth curve $s\mapsto\varphi(s)$ that is everywhere tangent to the vector field $v$. Under this philosophy, Poincar\'{e} made an exhaustive use of geometric methods to derive geometric properties of the solutions. Particularly, Poincar\'{e} set himself to study differential equations of the form 
\begin{equation}\label{preface:eq:1}
 \frac{dy}{dx}=\frac{P(x,y)}{Q(x,y)},
\end{equation}
where $(x,y)\in\R^2$ and $P,Q$ are real polynomials. To an autonomous equation such as (\ref{preface:eq:1}) we can associate a polynomial two-dimensional vector field $\displaystyle v=P(x,y)\frac{\partial}{\partial x} + Q(x,y)\frac{\partial}{\partial y}$. Poincar\'{e} himself introduced the key concept of a \textit{limit cycle}: a closed trajectory (i.e.~periodic solution) isolated from other such trajectories, and proved that a polynomial differential equation in the plane with no saddle connections may only have a finite number of limit cycles. 
%
%citation?
%
In 1900, during the International Congress of Mathematicians, Hilbert included in his famous list of problems the following question:

% \medskip
\begin{quote}
\singlespacing
 What can be said about the maximal number and position of Poincar\'{e} boundry cycles (\textit{limit cycles}) for a differential equation of the first order of the form $\frac{dy}{dx}=\frac{P}{Q}$, where $P$ and $Q$ are $n^{\text{th}}$ degree polynomials in $x$ and $y$?
\end{quote}

\vspace{-2\parskip} %yuk!
\noindent This question, which is the second part of Hilbert's $16^{\text{th}}$ problem, remains open--even in the simplest non-trivial case: that of quadratic polynomials.

The study of \textit{complex} vector fields goes back to Poincar\'{e} and Dulac, yet a huge development took place towards the end of the 1950's, when Russian mathematicians Petrovskii and Landis published a paper claiming to have a complete solution to Hilbert's $16^{\text{th}}$ problem \cite{PetrovskiiLandis1957}. The strategy in this paper is to extend equation (\ref{preface:eq:1}) to the complex domain. In such extension the solutions to the equation define, outside the equilibrium points
 \[ \Sigma = \set{(x,y)\in\C^2}{P(x,y)=Q(x,y)=0}, \]
complex analytic curves locally parametrized by a complex variable $t\in\C$. In this way, the solutions decompose $\C^2\setminus\Sigma$ into a disjoint union of curves. Moreover, by the flow-box theorem, this partition locally looks like the standard partition of the bidisk $\mathbb{D}\times\mathbb{D}$ by parallel horizontal slices $\mathbb{D}\times\{y\}$. These are precisely the defining properties of what we know as a \textit{singular holomorphic foliation}. The integral curves of the equation are called the \textit{leaves} of the foliation and $\Sigma$ its \textit{singular set}. Many of the concepts in the theory of real differential equations have a straightforward complex analog. In particular, the concept of a limit cycle can be naturally extended to the complex domain. The Petrovskii-Landis strategy consisted of bounding the number of complex limit cycles that may appear in the \textit{complexification} of equation (\ref{preface:eq:1}), thus obtaining a bound on the number of real limit cycles. 

In the middle of the 1960's Ilyashenko and Novikov found a crucial gap in the solution to Hilbert's $16^{\text{th}}$ problem, which completely invalidated the results claimed by Petrovskii and Landis. The history of this still-open problem has certainly been dramatic, yet it has inspired a great development in the geometric theory of differential equations and marked a start point for the study of complex polynomial foliations on $\C^2$.

A crucial further development of the theory was later achieved by Hudai-Verenov and Ilyashenko \cite{HudaiVerenov1962,Ilyashenko1978}, who proved that the \textit{generic} properties of complex polynomial foliations are strikingly different than those of real polynomial foliations. In particular, generic polynomial foliations have the following properties:
\begin{itemize}
 \item Rigidity: Topological equivalence is extremely rare and closely related to analytic equivalence.
 \item Density of solutions: Every leaf on $\C\setminus\Sigma$ is everywhere dense on $\C^2$.
 \item Infinite amount of limit cycles: There are countably many homologically independent complex limit cycles.
\end{itemize}
The meaning of the word ``generic'' above is something that has been refined thanks to many authors throughout the years (eg.~\cite{Shcherbakov1984,Nakai1994,LinsNetoSadScardua1998}) and is now taken to mean that there exists an open and dense set in the space of all polynomial foliations of a fixed degree (defined by either algebraic or analytic conditions) where every foliation from that set satisfies the above properties.

\bigskip
\noindent There are two main contributions of this thesis to the theory of polynomial vector fields on $\C^2$.

First, improving on previous results, we show that foliations defined by generic quadratic vector fields have the strongest imaginable rigidity property.

\begin{prefthm}
Two foliations defined by generic quadratic vector fields are topologically equivalent if and only if they are affine equivalent.  
\end{prefthm}

Second, we provide two completely independent results that provide necessary and sufficient conditions for generic quadratic vector fields to be affine equivalent. Previously, no results of this sort were know. %Double check this claim!! haha..

\begin{prefthm}
 Two generic quadratic vector fields are orbitally affine equivalent if and only if their holonomy groups at infinity are analytically conjugate.
\end{prefthm}

\begin{prefthm}
 Two generic quadratic vector fields are affine equivalent if and only if they have the same spectra of singularities and the same characteristic numbers at infinity.
\end{prefthm}

These results have been published in \cite{TwinVectorFields,UtmostRigidity} and will be discussed carefully in the present work.





% \chapter{Introduction}
% 
% \chapter{The holonomy group at infinity}
% 
% \chapter{The spectra of singularities}
% 
% \appendix
% \chapter{Chapter 1 of appendix}
% Appendix chapter 1 text goes here

% \nocite{StrongTopoInvariance,UtmostRigidity,TwinVectorFields,WoodsHole}
\bibliography{ref-valente}

\end{document}






















