\documentclass[phd,tocprelim]{cornell}
 \let\ifpdf\relax % to avoid name clash error
%
% tocprelim option must be included to put the roman numeral pages in the
% table of contents
%
% The cornellheadings option will make headings completely consistent with
% guidelines.
%
% This sample document was originally provided by Blake Jacquot, and
% fixed up by Andrew Myers.
%
% Some possible packages to include
% \usepackage{graphicx,pstricks}
% \usepackage{graphics}
% \usepackage{moreverb}
% \usepackage{subfigure}
% \usepackage{epsfig}
% \usepackage{subfigure}
% \usepackage{hangcaption}
% \usepackage{txfonts}
% \usepackage{palatino}

\usepackage{valente}
 \hypersetup{pdftitle={My Thesis},pdfauthor={Valente Ramirez}}

%if you're having problems with overfull boxes, you may need to increase
%the tolerance to 9999
%\tolerance=9999

\bibliographystyle{alpha} % What is required by the Graduate School?
% \bibliographystyle{IEEEbib}

\renewcommand{\caption}[1]{\singlespacing\hangcaption{#1}\normalspacing}
\renewcommand{\topfraction}{0.85}
\renewcommand{\textfraction}{0.1}
\renewcommand{\floatpagefraction}{0.75}

% Personal information
\title {Quadratic vector fields on \texorpdfstring{$\C^2$}{C2}\\rigidity and analytic invariants}
	% Warning: If the title is modified, make sure it matches the \@abstracttitle in cornell.cls
\author {Valente Ram\'{i}rez Garc\'{i}a Luna}
\conferraldate {August}{2017}
\degreefield {Ph.D.}
\copyrightholder{Valente Ram\'{i}rez Garc\'{i}a Luna}
\copyrightyear{2017}

\begin{document}

\maketitle
% \makecopyright

% Abstract
\cleardoublepage\phantomsection % Use this to make the index link to the right page
\begin{abstract}
 This work deals with \textit{generic} quadratic vector fields on $\C^2$ and the holomorphic foliations that these vector fields define on $\CP^2$. We assume that the extended foliation has non-degenerate singularities only and an invariant line at infinity.
 
 The first part of the present work deals with the \textit{extended spectra of singularities}. The extended spectra is the collection of the eigenvalues of the linearization of the vector field at each of the singular points in the affine part, together with the characteristic numbers (i.e.~Camacho-Sad indices) of the singularities on the line at infinity. We discuss what are the relations among these numbers that every generic quadratic vector field is bound to satisfy. Moreover, we conclude that two generic quadratic vector fields are affine equivalent if and only if they have the same extended spectra of singularities.
 
 In the second part we focus on the \textit{holonomy group} at infinity. We show that two generic quadratic vector fields that have conjugate holonomy groups must be orbitally affine equivalent. In particular, this proves that generic quadratic vector fields exhibit the utmost rigidity property: two such vector fields are orbitally topologically equivalent if and only if they are orbitally affine equivalent.

\begin{metadata} % Environment defined manually in valente.sty
 \keywords{Quadratic vector fields $\cdot$ holomorphic foliations $\cdot$ \mbox{topological rigidity $\cdot$} holonomy group at infinity $\cdot$ spectra of singularities $\cdot$ Euler-Jacobi formula}
 
 \MSC{37F75 $\cdot$ 32M25 $\cdot$ 32S65}
\end{metadata}
\end{abstract}


% Biographical sketch
% \cleardoublepage\phantomsection
% \begin{biosketch}
% Your biosketch goes here. Make sure it sits inside
% the brackets.
% \end{biosketch}


% Dedication
% \cleardoublepage\phantomsection
% \begin{dedication}
% Dedication
% \end{dedication}


% Acknowledgements
% \cleardoublepage\phantomsection
% \begin{acknowledgements}
% Your acknowledgements go here. Make sure it sits inside the brackets.
% \end{acknowledgements}


% Table of contents
\hypersetup{linkcolor=black} % Do I want the toc have colored links?
\cleardoublepage\phantomsection
\contentspage
\hypersetup{linkcolor=red}
% \tablelistpage
% \figurelistpage


% Preface
\cleardoublepage\phantomsection
\addcontentsline{toc}{section}{Preface}
\chapter*{Preface}

The results presented in this thesis belong to the theory of \textit{holomorphic foliations}. More precisely, to the study of polynomial vector fields on $\C^2$ and the holomorphic foliations that such vector fields define on $\CP^2$. The theory of holomorphic foliations is a rich area of mathematics where several other branches of mathematics come together: ordinary differential equations, one-dimensional complex dynamics, complex analytic and complex algebraic geometry, singularity theory, topology. 

The origins of the theory of holomorphic foliations trace back to the beginning of the \textit{qualitative theory} of ordinary differential equations, pioneered by Poincar\'{e} towards the end of the XIX$^{\text{th}}$ century. 
%
%cit: H. Poincare, "Mémoire sur les courbes définiés par une équation différentielle" J. de Math--a series of papers between 1881 and 1886. cf. https://www.encyclopediaofmath.org/index.php/Qualitative_theory_of_differential_equations
%
The brilliant insight of Poincar\'{e} was that differential equations belong not only to the realm of analysis, but also to geometry. For example, providing a non-autonomous differential equation of the form $\displaystyle\frac{dx}{dt}=f(x,t)$ is equivalent to giving a vector field, a much more geometric object, of the form
 \[ v = f(x,t)\frac{\partial}{\partial x} + \frac{\partial}{\partial t}, \qquad (x,t)\in U\subset\R^2, \]
in some appropriate open set $U$. A solution to the given differential equation corresponds to a smooth curve $s\mapsto\varphi(s)$ that is everywhere tangent to the vector field $v$. Under this philosophy, Poincar\'{e} made an exhaustive use of geometric methods to derive geometric properties of the solutions. Particularly, Poincar\'{e} set himself to study of systems of differential equations of the form 
\begin{equation}\label{preface:eq:1}
 \frac{dx}{dt}=P(x,y), \quad \frac{dy}{dt}=Q(x,y),
\end{equation}
where $(x,y)\in\R^2$ and $P,Q$ are real polynomials. To an autonomous equation such as (\ref{preface:eq:1}) we can associate a polynomial two-dimensional vector field $\displaystyle v=P(x,y)\frac{\partial}{\partial x} + Q(x,y)\frac{\partial}{\partial y}$. Poincar\'{e} himself introduced the key concept of a \textit{limit cycle}: a closed trajectory (i.e.~periodic solution) isolated from other such trajectories, and proved that a polynomial differential equation in the plane with no saddle connections may only have a finite number of limit cycles. 
%
%citation?
%
In 1900, during the International Congress of Mathematicians, Hilbert included in his famous list of problems the following question:

\bigskip
\begin{quote}
\singlespacing
 What can be said about the maximal number and position of Poincar\'{e} boundry cycles (\textit{cycles limites}) for a differential equation of the first order of the form $\frac{dy}{dx}=\frac{Q}{P}$, where $P$ and $Q$ are $n^{\text{th}}$ degree polynomials in $x$ and $y$?
\end{quote}

\noindent This question, which is the second part of Hilbert's $16^{\text{th}}$ problem, remains open--even in the simplest non-trivial case: that of quadratic polynomials.

The study of \textit{complex} vector fields goes back to Poincar\'{e} and Dulac, yet a huge development took place towards the end of the 1950's, when Russian mathematicians Petrovskii and Landis published a paper claiming to have a complete solution to Hilbert's $16^{\text{th}}$ problem \cite{PetrovskiiLandis1957}. The strategy in this paper is to extend equation (\ref{preface:eq:1}) to the complex domain. In such extension the solutions to the equation define, outside the set of equilibrium points
 \[ \Sigma = \set{(x,y)\in\C^2}{P(x,y)=Q(x,y)=0}, \]
complex analytic curves locally parametrized by a complex variable $t\in\C$. In this way, the solutions decompose $\C^2\setminus\Sigma$ into a disjoint union of curves. Moreover, by the flow-box theorem, this partition locally looks like the standard partition of the bidisk $\mathbb{D}\times\mathbb{D}$ by parallel horizontal slices $\mathbb{D}\times\{y\}$. These are precisely the defining properties of what we know as a \textit{singular holomorphic foliation}. The integral curves of the equation are called the \textit{leaves} of the foliation and $\Sigma$ its \textit{singular set}. Many of the concepts in the theory of real differential equations have a straightforward complex analog. In particular, the concept of a limit cycle can be naturally extended to the complex domain. The Petrovskii-Landis strategy consisted of bounding the number of complex limit cycles that may appear in the \textit{complexification} of equation (\ref{preface:eq:1}), thus obtaining a bound on the number of real limit cycles. 

In the middle of the 1960's Ilyashenko and Novikov found a crucial gap in the solution to Hilbert's $16^{\text{th}}$ problem, which completely invalidated the results claimed by Petrovskii and Landis. The history of this still-open problem has certainly been dramatic, yet it has inspired a great development in the geometric theory of differential equations and marked a start point for the study of complex polynomial foliations on $\C^2$.

A crucial further development of the theory was later achieved by Hudai-Verenov and Ilyashenko \cite{HudaiVerenov1962,Ilyashenko1978}, who proved that the \textit{generic} properties of complex polynomial foliations are strikingly different than those of real polynomial foliations. In particular, generic polynomial foliations have the following properties:
\begin{itemize}
 \item Rigidity: Topological equivalence is extremely rare and closely related to analytic equivalence.
 \item Density of solutions: Every leaf on $\C\setminus\Sigma$ is everywhere dense on $\C^2$.
 \item Infinite amount of limit cycles: There are countably many homologically independent complex limit cycles.
\end{itemize}
The meaning of the word ``generic'' above is something that has been refined thanks to many authors throughout the years (eg.~\cite{Shcherbakov1984,Nakai1994,LinsNetoSadScardua1998}) and is now taken to mean that there exists an open and dense set in the space of all polynomial foliations of a fixed degree (defined by either algebraic or analytic conditions) where every foliation from that set satisfies the above properties.

\bigskip
\noindent There are two main contributions of this thesis to the theory of polynomial vector fields on $\C^2$.

First, improving on previous results, we show that foliations defined by generic quadratic vector fields have the strongest imaginable rigidity property.

\begin{prefthm}
Two foliations defined by generic quadratic vector fields are topologically equivalent if and only if they are affine equivalent.  
\end{prefthm}

Second, we provide two completely independent results that provide necessary and sufficient conditions for generic quadratic vector fields to be affine equivalent. Previously, no results of this sort were known. %Double check this claim!! haha..

\begin{prefthm}
 Two generic quadratic vector fields are orbitally affine equivalent if and only if their holonomy groups at infinity are analytically conjugate.
\end{prefthm}

\begin{prefthm}
 Two generic quadratic vector fields are affine equivalent if and only if they have the same spectra of singularities and the same characteristic numbers at infinity.
\end{prefthm}

These results have been published in \cite{TwinVectorFields,UtmostRigidity} and will be discussed in detail in the present work.


% \cleardoublepage
% \normalspacing \setcounter{page}{1} \pagenumbering{arabic}
\pagestyle{cornell} \addtolength{\parskip}{0.5\baselineskip}

\numberwithin{equation}{section}




\chapter{Introduction}

\section{The class of generic quadratic vector fields}

The object of study in this work are \textit{quadratic vector fields} on $\C^2$. These are sections of the holomorphic tangent bundle of $\C^2$ of the form 
 \[ v = P(x,y)\frac{\partial}{\partial x} + Q(x,y)\frac{\partial}{\partial y}, \]
where $P,Q\in\C[x,y]$ are polynomials of degree two. Let us regard $\C^2$ as an affine chart on $\CP^2$. A vector field $v$ as above does not have a holomorphic extension to $\CP^2$, since it is known that any holomorphic vector field on $\CP^2$ comes from a linear homogeneous vector field on $\C^3$. Nonetheless, the \textit{holomorphic foliation} that the integral curves of $v$ define on $\C^2$ can always be extended to a singular holomorphic foliation $\F_v$ on $\CP^2$. Such extension is possible for any polynomial vector field on $\C^2$. Conversely, any singular holomorphic foliation on $\CP^2$ is induced by a polynomial vector field on any affine chart.

The space of polynomial vector fields of a fixed degree has a natural vector space structure. We seek to understand the \textit{generic properties} of such vector fields. We will say that a property is generic if it is satisfied by every vector field in some open and dense subset of the space of all vector fields of the given degree. The nature of these open and dense sets will be specified when needed. For example, by Bezout's theorem, a generic vector field of degree $n$ has exactly $n^2$ isolated singularities. Also, it is well known that when extending a foliation from $\C^2$ to $\CP^2$ we generically obtain an invariant line at infinity. This means that the line $\mathcal{L}=\CP^2\setminus\C^2$, once a finite number of singularities are removed, is a leaf of the foliation $\F_v$. In the generic case, the number of singularities on $\mathcal{L}$ is exactly $n+1$. These conditions are the most basic generic properties of polynomial vector fields, they are all defined by algebraic conditions and, we will deal exclusively with quadratic vector fields with these properties. 

\begin{definition}\label{def:classesV2A2}
 We will denote by $\V_2$ the space of all quadratic vector fields $v$ on $\C^2$ having exactly 4 isolated singularities, and such that $\F_v$ has an invariant line at infinity with exactly 3 singular points. Also, we will denote by $\A_2$ the space of all \textit{foliations} that come from a vector field from the class $\V_2$.
\end{definition}

Note that two quadratic vector fields generate the same foliation if and only if they differ only by a non-zero scaling factor; hence, $\A_2=\V_2/\C^*$.




\section{Equivalence of vector fields and foliations}

The properties we aim to study should not only be generic, but should also be preserved under  analytic (or topological) equivalence. We consider three types of relations: equivalence of vector fields over $\C^2$, orbital equivalence of vector fields over $\C^2$, equivalence of foliations over $\CP^2$. The following definitions are standard.

\begin{definition}\label{def:analyticEqVectorFields}
 We say that two vector fields $v_1$ and $v_2$ are \textit{analytically equivalent} if there exists an analytic diffeomorphism $F$ 
 %between the respective ambient spaces 
 that transforms one into the other; that is, $v_2(p)=DF(F^{-1}(p))\cdot v_1(F^{-1}(p))$. In case the diffeomorphism $F$ is an affine transformation on $\C^2$, we will say that the vector fields are \textit{affine equivalent}. 
\end{definition}

\begin{definition}\label{def:analyticEqFoliations}
 Let $\F_1$ and $\F_2$ be singular holomorphic foliations on a complex manifold $M$. We say that $\F_1$ and $\F_2$ are \textit{topologically equivalent} if there exists a homeomorphism  of $M$ which brings the leaves of $\F_1$ onto the leaves of $\F_2$ while preserving both the orientation of the leaves and of ambient space $M$. If such homeomorphism is in fact an analytic diffeomorphism, we will say that the foliations are \textit{analytically equivalent}.
\end{definition}

\begin{definition}\label{def:orbitalEq}
 Two vector fields on $\C^2$ will be said to be \textit{orbitally} equivalent if the singular holomorphic foliations they define are equivalent over $\C^2$ in the respective sense (topological or analytic).
\end{definition}




\section{Analytic invariants of polynomial vector fields}

By \textit{analytic invariants} of a vector field we mean those \textit{objects} that we can coherently associate to each (say generic) vector field which are preserved under analytic equivalence of vector fields. These objects may be numbers, groups, cohomology classes, sheaves, and so on. Good analytic invariants should give us valuable information about the behavior of our vector field. Moreover, these invariants are fundamental in the understanding of the \textit{analytic classification} of vector fields: in order for two vector fields to be analytically equivalent, it is necessary that their invariants should agree. The question of whether or not coincidence of the invariants is also sufficient is, in most cases, a very delicate question. 

Analytic invariants can be of a different nature: local, semi-local or global. For example, the \textit{projective degree} of a foliation (the number of tangencies with a generic line) or the set of invariant algebraic curves are global analytic invariants. For the class of generic quadratic vector fields these invariants are not interesting: a quadratic vector field with an invariant line at infinity has always projective degree two, and in the generic case no other invariant curves. The analytic invariants we are interested here are of a local nature: the spectrum of the singularities; and semi-local: the holonomy group at infinity.


\subsection{The spectra of singularities}

Let $p$ be an isolated singular point of some vector field $v = P(x,y)\frac{\partial}{\partial x} + Q(x,y)\frac{\partial}{\partial y}$. Consider the \textit{linearization matrix}
 \[ Dv(p) = 
 \singlespacing
  \begin{pmatrix} 
   P'_x & P'_y \\
   Q'_x & Q'_y \\
  \end{pmatrix}\bigg\vert_{(x,y)=p\,.}  
   \]
It is well known that analytically equivalent vector fields have conjugate linearization matrices, hence the spectrum of the linearization matrix at each singular point is an analytic invariant.

\begin{definition}
 Let $p$ be a singular point of $v$. We define the \textit{spectrum} of $v$ at $p$ as the ordered pair $\operatorname{Spec}\,(v,p) = (\tr Dv(p),\det Dv(p))$. The \textit{spectrum of singularities} of $v$ is the set 
  \[ \singlespacing \operatorname{Spec}\,v = \set{\operatorname{Spec}\,(v,p)}{v(p)=0} . \]
\end{definition}


\subsection{Characteristic numbers}


\subsection{The holonomy group}






% \section{Statement of the results}
% \subsection{Moduli of analytic classification}
% \subsection{Twin vector fields and relations on spectra}
% \subsection{The phenomenon of topological rigidity}




% \chapter{The spectra of singularities}
% 
% \chapter{The holonomy group at infinity}
% 
% \appendix
% \chapter{Chapter 1 of appendix}
% Appendix chapter 1 text goes here

% \nocite{StrongTopoInvariance,UtmostRigidity,TwinVectorFields,WoodsHole}
\bibliography{ref-valente}

\end{document}







%%COMMENTS%%

% Do I want to switch CP2 by P2?
% I introduced the definition ``Spec p'' as Spec Dv(p). Make sure it is used in Chapter 2.
% Excesive amount of space in display math mode can be fixed with \singlespacing.
% Check the use of ``extended spectra''.














