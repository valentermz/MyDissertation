\documentclass[phd,tocprelim]{cornell}
 \let\ifpdf\relax % to avoid name clash error
%
% tocprelim option must be included to put the roman numeral pages in the
% table of contents
%
% The cornellheadings option will make headings completely consistent with
% guidelines.
%
% This sample document was originally provided by Blake Jacquot, and
% fixed up by Andrew Myers.
%
% Some possible packages to include
% \usepackage{graphicx,pstricks}
% \usepackage{graphics}
% \usepackage{moreverb}
% \usepackage{subfigure}
% \usepackage{epsfig}
% \usepackage{subfigure}
% \usepackage{hangcaption}
% \usepackage{txfonts}
% \usepackage{palatino}

\usepackage{valente}
 \hypersetup{pdftitle={My Thesis},pdfauthor={Valente Ramirez}}

%if you're having problems with overfull boxes, you may need to increase
%the tolerance to 9999
%\tolerance=9999

\bibliographystyle{plain}
% \bibliographystyle{IEEEbib}

\renewcommand{\caption}[1]{\singlespacing\hangcaption{#1}\normalspacing}
\renewcommand{\topfraction}{0.85}
\renewcommand{\textfraction}{0.1}
\renewcommand{\floatpagefraction}{0.75}

% Personal information
\title {The analytic invariants of generic quadratic vector fields on \texorpdfstring{$\C^2$}{C2}}
	% Warning: If the title is modified, make sure it matches the \@abstracttitle in cornell.cls
\author {Valente Ram\'{i}rez Garc\'{i}a Luna}
\conferraldate {August}{2017}
\degreefield {Ph.D.}
\copyrightholder{Valente Ram\'{i}rez Garc\'{i}a Luna}
\copyrightyear{2017}

\begin{document}

\maketitle
\makecopyright

\begin{abstract}
Your abstract goes here. Make sure it sits inside the brackets. If not,
your biosketch page may not be roman numeral iii, as required by the
graduate school.
\end{abstract}

\begin{biosketch}
Your biosketch goes here. Make sure it sits inside
the brackets.
\end{biosketch}

\begin{dedication}
Dedication
\end{dedication}

\begin{acknowledgements}
Your acknowledgements go here. Make sure it sits inside the brackets.
\end{acknowledgements}

\contentspage
% \tablelistpage
% \figurelistpage

\normalspacing \setcounter{page}{1} \pagenumbering{arabic}
\pagestyle{cornell} \addtolength{\parskip}{0.5\baselineskip}


\addcontentsline{toc}{chapter}{Preface}
\chapter*{Preface}

The results presented in this thesis belong to the theory of \textit{holomorphic foliations}. More precisely, to the study of polynomial vector fields on $\C^2$ and the holomorphic foliations that such vector fields define on $\CP^2$. The theory of holomorphic foliations is a rich area of mathematics where several other branches of mathematics come together: ordinary differential equations, one-dimensional complex dynamics, complex analytic and complex algebraic geometry, singularity theory, topology. 

The origins of the theory of holomorphic foliations trace back to the beginning of the \textit{qualitative theory} of ordinary differential equations, pioneered by Poincar\'{e} towards the end of the XIX$^{\text{th}}$ century. 
%
%cit: H. Poincaré, "Mémoire sur les courbes définiés par une équation différentielle" J. de Math--a series of papers between 1881 and 1886. cf. https://www.encyclopediaofmath.org/index.php/Qualitative_theory_of_differential_equations
%
The brilliant insight of Poincar\'{e} was that differential equations belong not only to the realm of analysis, but also to geometry. For example, providing a non-autonomous differential equation of the form $\displaystyle\frac{dx}{dt}=f(x,t)$ is equivalent to giving a vector field, a much more geometric object, of the form
 \[ v = f(x,t)\frac{\partial}{\partial x} + \frac{\partial}{\partial t}, \qquad (x,t)\in U\subset\R^2, \]
in some appropriate open set $U$. A solution to the given differetial equation now corresponds to a smooth curve $s\mapsto\varphi(s)$ that is everywhere tangent to the vector field $v$. Under this philosophy, Poincar\'{e} made an exhaustive use of geometric methods to derive geometric properties of the solutions. Particularly, Poincar\'{e} set himself to study differential equations of the form 
\begin{equation}\label{preface:eq:1}
 \frac{dy}{dx}=\frac{P(x,y)}{Q(x,y)},
\end{equation}
where $(x,y)\in\R^2$ and $P,Q$ are real polynomials. To an autonomous equation such as (\ref{preface:eq:1}) we can associate a polynomial two-dimensional vector field $\displaystyle v=P(x,y)\frac{\partial}{\partial x} + Q(x,y)\frac{\partial}{\partial y}$. Poincar\'{e} himself introduced the key concept of a \textit{limit cycle}: a closed trajectory (i.e.~periodic solution) isolated from other such trajectories, and proved that a polynomial differential equation in the plane with no saddle connections may only have a finite number of limit cycles. 
%
%citation?
%
In 1900 Hilbert included in his famous list of problems the following question:

% \medskip
\begin{quote}
\singlespacing
 What can be said about the maximal number and position of Poincar\'{e} boundry cycles (\textit{limit cycles}) for a differential equation of the first order of the form $\frac{dy}{dx}=\frac{P}{Q}$, where $P$ and $Q$ are $n^{\text{th}}$ degree polynomials in $x$ and $y$?
\end{quote}

\vspace{-2\parskip} %yuk!
\noindent This question, which is the second part of Hilbert's 16$^{\text{th}}$ problem, remains open--even in the simplest non-trivial case: that of quadratic polynomials.







\chapter{Introduction}

\chapter{The holonomy group at infinity}

\chapter{The spectra of singularities}

% \appendix
% \chapter{Chapter 1 of appendix}
% Appendix chapter 1 text goes here

\nocite{StrongTopoInvariance,UtmostRigidity,TwinVectorFields,WoodsHole}
\bibliography{ref-valente}

\end{document}






















