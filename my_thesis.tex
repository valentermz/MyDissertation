\documentclass[phd,tocprelim]{cornell}
 \let\ifpdf\relax % to avoid name clash error
%
% tocprelim option must be included to put the roman numeral pages in the
% table of contents
%
% The cornellheadings option will make headings completely consistent with
% guidelines.
%
% This sample document was originally provided by Blake Jacquot, and
% fixed up by Andrew Myers.
%
% Some possible packages to include
% \usepackage{graphicx,pstricks}
% \usepackage{graphics}
% \usepackage{moreverb}
% \usepackage{subfigure}
% \usepackage{epsfig}
% \usepackage{subfigure}
% \usepackage{hangcaption}
% \usepackage{txfonts}
% \usepackage{palatino}

\usepackage{valente}
 \hypersetup{pdftitle={My Thesis},pdfauthor={Valente Ramirez}}

%if you're having problems with overfull boxes, you may need to increase
%the tolerance to 9999
%\tolerance=9999

\bibliographystyle{alpha} % What is required by the Graduate School?
% \bibliographystyle{IEEEbib}

\renewcommand{\caption}[1]{\singlespacing\hangcaption{#1}\normalspacing}
\renewcommand{\topfraction}{0.85}
\renewcommand{\textfraction}{0.1}
\renewcommand{\floatpagefraction}{0.75}

% Personal information
\title {Quadratic vector fields on \texorpdfstring{$\C^2$}{C2}\\rigidity and analytic invariants}
	% Warning: If the title is modified, make sure it matches the \@abstracttitle in cornell.cls
\author {Valente Ram\'{i}rez Garc\'{i}a Luna}
\conferraldate {August}{2017}
\degreefield {Ph.D.}
\copyrightholder{Valente Ram\'{i}rez Garc\'{i}a Luna}
\copyrightyear{2017}

\begin{document}

\maketitle
% \makecopyright

% Abstract
\cleardoublepage\phantomsection % Use this to make the index link to the right page
\begin{abstract}
 This work deals with \textit{generic} quadratic vector fields on $\C^2$ and the holomorphic foliations that these vector fields define on $\CP^2$. We assume that the extended foliation has non-degenerate singularities only and an invariant line at infinity.
 
 The first part of the present work deals with the \textit{extended spectra of singularities}. The extended spectra is the collection of the eigenvalues of the linearization of the vector field at each of the singular points in the affine part, together with the characteristic numbers (i.e.~Camacho-Sad indices) of the singularities on the line at infinity. We discuss what are the relations among these numbers that every generic quadratic vector field is bound to satisfy. Moreover, we conclude that two generic quadratic vector fields are affine equivalent if and only if they have the same extended spectra of singularities.
 
 In the second part we focus on the \textit{holonomy group} at infinity. We show that two generic quadratic vector fields that have conjugate holonomy groups must be orbitally affine equivalent. In particular, this proves that generic quadratic vector fields exhibit the utmost rigidity property: two such vector fields are orbitally topologically equivalent if and only if they are orbitally affine equivalent.

\begin{metadata} % Environment defined manually in valente.sty
 \keywords{Quadratic vector fields $\cdot$ holomorphic foliations $\cdot$ \mbox{topological rigidity $\cdot$} holonomy group at infinity $\cdot$ spectra of singularities $\cdot$ Euler-Jacobi formula}
 
 \MSC{37F75 $\cdot$ 32M25 $\cdot$ 32S65}
\end{metadata}
\end{abstract}


% Biographical sketch
% \cleardoublepage\phantomsection
% \begin{biosketch}
% Your biosketch goes here. Make sure it sits inside
% the brackets.
% \end{biosketch}


% Dedication
% \cleardoublepage\phantomsection
% \begin{dedication}
% Dedication
% \end{dedication}


% Acknowledgements
% \cleardoublepage\phantomsection
% \begin{acknowledgements}
% Your acknowledgements go here. Make sure it sits inside the brackets.
% \end{acknowledgements}


% Table of contents
\hypersetup{linkcolor=black} % Do I want the toc have colored links?
\cleardoublepage\phantomsection
\contentspage
\hypersetup{linkcolor=red}
% \tablelistpage
% \figurelistpage


% Preface
\cleardoublepage\phantomsection
\addcontentsline{toc}{section}{Preface}
\chapter*{Preface}

The results presented in this thesis belong to the theory of \textit{holomorphic foliations}. More precisely, to the study of polynomial vector fields on $\C^2$ and the holomorphic foliations that such vector fields define on $\CP^2$. The theory of holomorphic foliations is a rich area of mathematics where several other branches of mathematics come together: ordinary differential equations, one-dimensional complex dynamics, complex analytic and complex algebraic geometry, singularity theory, topology. 

The origins of the theory of holomorphic foliations trace back to the beginning of the \textit{qualitative theory} of ordinary differential equations, pioneered by Poincar\'{e} towards the end of the XIX$^{\text{th}}$ century. 
%
%cit: H. Poincare, "Mémoire sur les courbes définiés par une équation différentielle" J. de Math--a series of papers between 1881 and 1886. cf. https://www.encyclopediaofmath.org/index.php/Qualitative_theory_of_differential_equations
%
The brilliant insight of Poincar\'{e} was that differential equations belong not only to the realm of analysis, but also to geometry. For example, providing a non-autonomous differential equation of the form $\displaystyle\frac{dx}{dt}=f(x,t)$ is equivalent to giving a vector field, a much more geometric object, of the form
 \[ v = f(x,t)\frac{\partial}{\partial x} + \frac{\partial}{\partial t}, \qquad (x,t)\in U\subset\R^2, \]
in some appropriate open set $U$. A solution to the given differential equation corresponds to a smooth curve $s\mapsto\varphi(s)$ that is everywhere tangent to the vector field $v$. Under this philosophy, Poincar\'{e} made an exhaustive use of geometric methods to derive geometric properties of the solutions. Particularly, Poincar\'{e} set himself to study of systems of differential equations of the form 
\begin{equation}\label{preface:eq:1}
 \frac{dx}{dt}=P(x,y), \quad \frac{dy}{dt}=Q(x,y),
\end{equation}
where $(x,y)\in\R^2$ and $P,Q$ are real polynomials. To an autonomous equation such as (\ref{preface:eq:1}) we can associate a polynomial two-dimensional vector field $\displaystyle v=P(x,y)\frac{\partial}{\partial x} + Q(x,y)\frac{\partial}{\partial y}$. Poincar\'{e} himself introduced the key concept of a \textit{limit cycle}: a closed trajectory (i.e.~periodic solution) isolated from other such trajectories, and proved that a polynomial differential equation in the plane with no saddle connections may only have a finite number of limit cycles. 
%
%citation?
%
In 1900, during the International Congress of Mathematicians, Hilbert included in his famous list of problems the following question:

\bigskip
\begin{quote}
\singlespacing
 What can be said about the maximal number and position of Poincar\'{e} boundry cycles (\textit{cycles limites}) for a differential equation of the first order of the form $\frac{dy}{dx}=\frac{Q}{P}$, where $P$ and $Q$ are $n^{\text{th}}$ degree polynomials in $x$ and $y$?
\end{quote}

\noindent This question, which is the second part of Hilbert's $16^{\text{th}}$ problem, remains open--even in the simplest non-trivial case: that of quadratic polynomials.

The study of \textit{complex} vector fields goes back to Poincar\'{e} and Dulac, yet a huge development took place towards the end of the 1950's, when Russian mathematicians Petrovskii and Landis published a paper claiming to have a complete solution to Hilbert's $16^{\text{th}}$ problem \cite{PetrovskiiLandis1957}. The strategy in this paper is to extend equation (\ref{preface:eq:1}) to the complex domain. In such extension the solutions to the equation define, outside the set of equilibrium points
 \[ \Sigma = \set{(x,y)\in\C^2}{P(x,y)=Q(x,y)=0}, \]
complex analytic curves locally parametrized by a complex variable $t\in\C$. In this way, the solutions decompose $\C^2\setminus\Sigma$ into a disjoint union of curves. Moreover, by the flow-box theorem, this partition locally looks like the standard partition of the bidisk $\mathbb{D}\times\mathbb{D}$ by parallel horizontal slices $\mathbb{D}\times\{y\}$. These are precisely the defining properties of what we know as a \textit{singular holomorphic foliation}. The integral curves of the equation are called the \textit{leaves} of the foliation and $\Sigma$ its \textit{singular set}. Many of the concepts in the theory of real differential equations have a straightforward complex analog. In particular, the concept of a limit cycle can be naturally extended to the complex domain. The Petrovskii-Landis strategy consisted of bounding the number of complex limit cycles that may appear in the \textit{complexification} of equation (\ref{preface:eq:1}), thus obtaining a bound on the number of real limit cycles. 

In the middle of the 1960's Ilyashenko and Novikov found a crucial gap in the solution to Hilbert's $16^{\text{th}}$ problem, which completely invalidated the results claimed by Petrovskii and Landis. The history of this still-open problem has certainly been dramatic, yet it has inspired a great development in the geometric theory of differential equations and marked a start point for the study of complex polynomial foliations on $\C^2$.

A crucial further development of the theory was later achieved by Hudai-Verenov and Ilyashenko \cite{HudaiVerenov1962,Ilyashenko1978}, who proved that the \textit{generic} properties of complex polynomial foliations are strikingly different than those of real polynomial foliations. In particular, generic polynomial foliations have the following properties:
\begin{itemize}[topsep=0pt,noitemsep]
 \item Rigidity: Topological equivalence is extremely rare and closely related to analytic equivalence.
 \item Density of solutions: Every leaf on $\C\setminus\Sigma$ is everywhere dense on $\C^2$.
 \item Infinite amount of limit cycles: There are countably many homologically independent complex limit cycles.
\end{itemize}
The meaning of the word ``generic'' above is something that has been refined thanks to many authors throughout the years (eg.~\cite{Shcherbakov1984,Nakai1994,LinsNetoSadScardua1998}) and is now taken to mean that there exists an open and dense set in the space of all polynomial foliations of a fixed degree (defined by either algebraic or analytic conditions) where every foliation from that set satisfies the above properties.

\bigskip
\noindent There are two main contributions of this thesis to the theory of polynomial vector fields on $\C^2$.

First, we provide two completely independent results that provide necessary and sufficient conditions for generic quadratic vector fields to be affine equivalent. Previously, no results of this sort were known. 

\begin{prefthm}
 Two generic quadratic vector fields are affine equivalent if and only if they have the same spectra of singularities and the same characteristic numbers at infinity.
\end{prefthm}

\begin{prefthm}
 Two generic quadratic vector fields are orbitally affine equivalent if and only if their holonomy groups at infinity are analytically conjugate.
\end{prefthm}

Second, the fact that Theorem B implies that foliations defined by generic quadratic vector fields have the strongest imaginable rigidity property. This next theorem significantly improves, in the quadratic case, the previously known results about topological rigidity of polynomial foliations.

\begin{prefthm}
Two foliations defined by generic quadratic vector fields are topologically equivalent if and only if they are affine equivalent.  
\end{prefthm}  

These results have been published in \cite{TwinVectorFields,UtmostRigidity} and will be discussed in detail in the present work.


% \cleardoublepage
% \normalspacing \setcounter{page}{1} \pagenumbering{arabic}
\pagestyle{cornell} \addtolength{\parskip}{0.5\baselineskip}

\numberwithin{equation}{section}




\chapter{Introduction}

\section{The class of generic quadratic vector fields}

The object of study in this work are \textit{quadratic vector fields} on $\C^2$. These are sections of the holomorphic tangent bundle of $\C^2$ of the form 
 \[ v = P(x,y)\frac{\partial}{\partial x} + Q(x,y)\frac{\partial}{\partial y}, \]
where $P,Q\in\C[x,y]$ are polynomials of degree two. Let us regard $\C^2$ as an affine chart on $\CP^2$. In general, a vector field as above does not have a holomorphic extension to $\CP^2$, for it is known that any holomorphic vector field on $\CP^2$ comes from a linear homogeneous vector field on $\C^3$. Nonetheless, the \textit{holomorphic foliation} that the integral curves of $v$ define on $\C^2$ can always be extended to a singular holomorphic foliation $\F_v$ on $\CP^2$. Such extension is possible for any polynomial vector field on $\C^2$. And conversely, any singular holomorphic foliation on $\CP^2$ is induced by a polynomial vector field on any affine chart.

The space of polynomial vector fields of a fixed degree has a natural vector space structure. We seek to understand the \textit{generic properties} of such vector fields. We will say that a property is generic if it is satisfied by every vector field in some open and dense subset of the space of all vector fields of the given degree. The nature of these open and dense sets will be specified when needed. For example, by Bezout's theorem, a generic vector field of degree $n$ has exactly $n^2$ isolated singularities. Also, it is well known that when extending a foliation from $\C^2$ to $\CP^2$ we generically obtain an invariant line at infinity. This means that the line $\mathcal{L}=\CP^2\setminus\C^2$, once a finite number of singularities are removed, is a leaf of the foliation $\F_v$. In the generic case, the number of singularities on $\mathcal{L}$ is exactly $n+1$. These conditions are the most basic generic properties of polynomial vector fields, they are all defined by algebraic conditions and, we will deal exclusively with quadratic vector fields with these properties. 

\begin{definition}\label{def:classesV2A2}
 We will denote by $\V_2$ the space of all quadratic vector fields $v$ on $\C^2$ having exactly 4 isolated singularities, and such that $\F_v$ has an invariant line at infinity with exactly 3 singular points. Also, we will denote by $\A_2$ the space of all \textit{foliations} that come from a vector field from the class $\V_2$.
\end{definition}

\begin{remark}\label{rmk:scalingFactor}
 Two polynomial vector fields of equal degree generate the same foliation if and only if they differ only by a non-zero scaling factor; hence, $\A_2$ is given by the quotient $\V_2/\C\mbox{*}$.
\end{remark}




\section{Equivalence of vector fields and foliations}

The properties we aim to study should not only be generic, but should also be preserved under  analytic (or topological) equivalence. We consider three types of relations: equivalence of vector fields over $\C^2$, orbital equivalence of vector fields over $\C^2$, equivalence of foliations over $\CP^2$. The following definitions are standard.

\begin{definition}\label{def:analyticEqVectorFields}
 We say that two vector fields $v_1$ and $v_2$ are \textit{analytically equivalent} if there exists an analytic diffeomorphism $F$ 
 %between the respective ambient spaces 
 that transforms one into the other; that is, $v_2(p)=DF(F^{-1}(p))\cdot v_1(F^{-1}(p))$. In case the diffeomorphism $F$ is an affine transformation on $\C^2$, we will say that the vector fields are \textit{affine equivalent}. 
\end{definition}

\begin{definition}\label{def:analyticEqFoliations}
 Let $\F_1$ and $\F_2$ be singular holomorphic foliations on a complex manifold $M$. We say that $\F_1$ and $\F_2$ are \textit{topologically equivalent} if there exists a homeomorphism  of $M$ which brings the leaves of $\F_1$ onto the leaves of $\F_2$ while preserving both the orientation of the leaves and of ambient space $M$. If such homeomorphism is in fact an analytic diffeomorphism, we will say that the foliations are \textit{analytically equivalent}.
\end{definition}

\begin{definition}\label{def:orbitalEq}
 Two vector fields on $\C^2$ will be said to be \textit{orbitally} equivalent if the singular holomorphic foliations they define are equivalent over $\C^2$ in the respective sense (topological or analytic).
\end{definition}




\vspace{-1.5\parskip} %yuk!
\section{Analytic invariants of polynomial vector fields}\label{sec:analyticInvariants}

By \textit{analytic invariants} of a vector field we mean those \textit{objects} that we can coherently associate to each (say generic) vector field which are preserved under analytic equivalence. These objects may be numbers, groups, cohomology classes, sheaves, and so on. Good analytic invariants should give us valuable information about the behavior of the vector field in question. Moreover, these invariants are fundamental in the understanding of the \textit{analytic classification} of vector fields: in order for two vector fields to be analytically equivalent, it is necessary that their invariants should coincide. The question of whether or not coincidence of the invariants is also sufficient is, in most cases, a delicate question. 

Analytic invariants can be of a different nature: local, semi-local or global. For example, the \textit{projective degree} of a foliation (the number of tangencies with a generic line) or the set of invariant algebraic curves are global analytic invariants. For the class of generic quadratic vector fields these invariants are not interesting: a quadratic vector field with an invariant line at infinity has always projective degree two, and in the generic case no other invariant curves. The analytic invariants we are interested here are of a local nature: the spectra of the singularities; and of a semi-local nature: the holonomy group at infinity.


\subsection{The spectra of singularities}

Let $p$ be an isolated singular point of some vector field $v = P(x,y)\frac{\partial}{\partial x} + Q(x,y)\frac{\partial}{\partial y}$. Consider the \textit{linearization matrix}
 \[ Dv(p) = 
 \renewcommand*{\arraystretch}{0.65}
  \begin{pmatrix} 
   P'_x & P'_y \\
   Q'_x & Q'_y \\
  \end{pmatrix}\bigg\vert_{(x,y)=p\,.}  
   \]
It is well known that analytically equivalent vector fields have conjugate linearization matrices, hence the spectrum of the linearization matrix at each singular point is an analytic invariant. In case the ratio of the eigenvalues is not a real number the singularity is said to be \textit{hyperbolic}. It is well known that hyperbolic singularities are analytically linearizable. In this case, which is generic, the spectrum completely determines the local behavior of $v$ around $p$.

\begin{definition}
 Let $p$ be a singular point of $v$. We define the \textit{spectrum} of $v$ at $p$ as the ordered pair $\operatorname{Spec}(v,p) = (\tr Dv(p),\det Dv(p))$. The \textit{spectrum of singularities} of $v$ is the set 
  \[ \operatorname{Spec}\,v = \set{\operatorname{Spec}(v,p)}{v(p)=0} . \]
\end{definition}


\subsection{Characteristic numbers}

In order to study the extended foliation in a neighborhood of the line at infinity we introduce the following change of coordinates: $z=\displaystyle\frac{1}{x}$, $w=\displaystyle\frac{y}{x}$. A simple computation shows that, in these coordinates, a generic quadratic vector field induces a foliation given by an equation of the form
 \[ \frac{dz}{dw} = z\sum_{j=1}^3 \frac{\lambda_j}{w-w_j} + O(z^2). \]
The line at infinity is given by $\mathcal{L}=\{z=0\}$, and the singular point on it correspond to the poles $w_j$. The \textit{characteristic numbers at infinity} are defined to be the residues $\lambda_j$, which are precisely the Camacho-Sad indices $\lambda_j=\operatorname{CS}\,(\F_v,\mathcal{L},w_j)$. This number depends only on the local behavior of $\F_v$ around $w_j$, and in the hyperbolic case $(\lambda_j\notin\R)$ it determines it completely.

\begin{definition}\label{def:extendedSpectra}
 The \textit{extended spectra of singularities} of a generic quadratic vector field $v$ is the collection of the spectra of singularities over the affine part, together with the characteristic numbers at infinity.
\end{definition}


\subsection{The holonomy group}

Let $\F$ be a foliation from the class $\A_2$ and denote by $\LF$ its leaf at infinity (i.e.~the line punctured at the singularities). Given a base point $b\in\LF$ and the germ of a cross section $(\Gamma,b)$, transversal to the leaves of $\F$, we can perform the following standard construction: for any loop $\gamma$ on $\LF$ based at $b$, we follow the leaves of $\F$ along some small tubular neighborhood of the image of $\gamma$ to obtain the germ of a holomorphic return map $\Delta_\gamma\colon (\Gamma,b)\to(\Gamma,b)$. This germ, the \textit{holonomy} of $\F$ along $\gamma$, depends only on the homotopy class of $\gamma$ in the fundamental group $\pi_1(\LF,b)$. Moreover, the map $\Delta\colon\pi_1(\LF,b)\to \operatorname{Diff}\,(\Gamma,b)$ is a group anti-homomorphism. Fixing an analytic parametrization $\Diff\to \operatorname{Diff}\,(\Gamma,b)$, we will write $\Delta\colon\pi_1(\LF,b)\to\Diff$ and define the \textit{holonomy group at infinity} as $\GF=\operatorname{Im}\,\Delta\subset\Diff$. This group is canonically defined up to conjugacy in $\Diff$.

\begin{definition}
 We say that two foliations $\F$ and $\Ft$ from the class $\A_2$ have \textit{analytically conjugate holonomy groups} whenever there exist the germ of a conformal map $h$, and a geometric isomorphism $H_{\ast}\colon \pi_1(\LF,b)\to\pi_1(\LFt,\tilde{b})$ that conjugate the holonomy in the following way:
  \[ \forall\gamma\in\pi_1(\LF,b): \quad %
     h\circ\Delta_\gamma = \widetilde{\Delta}_{H_{\ast}\gamma}\circ h. \]
By a \textit{geometric isomorphism} we mean an isomorphism $H_{\ast}\colon \pi_1(\LF,b)\to\pi_1(\LFt,\tilde{b})$ induced by some orientation-preserving homeomorphism $H\colon\LF\to\LFt$.
\end{definition}

The holonomy group at infinity has been the main tool in the study of generic polynomial foliations. It is known that a generic polynomial foliation on $\C^2$ has a holonomy group at infinity that is \textit{non-solvable}. Under the assumption of non-solvability and a few extra mild restrictions, the geometric properties of the holonomy group determine to a great extent the geometric properties of the foliation on a global level. A finitely generated non-solvable (pseudo) group of conformal germs $\Diff\to\Diff$ satisfies the following properties:
%
\begin{itemize}[topsep=-2pt,itemsep=-4pt]
 \item The group is \textit{topologically rigid} \cite{Shcherbakov1984, Nakai1994},
 \item The orbits of the action are \textit{dense in sectors} \cite{Nakai1994},
 \item The group has a countable number of germs whose representatives have \textit{isolated fixed points} different from zero \cite{BelliartLiousseLoray1997, ShcherbakovRosalesOrtiz1998}.
\end{itemize}
%
In consequence, a generic polynomial foliation satisfies the following:
%
\begin{itemize}[topsep=-2pt,itemsep=-4pt]
 \item The foliation is \textit{absolutely rigid}  \cite{Ilyashenko1978, Shcherbakov1984, LinsNetoSadScardua1998},
 \item The leaves of the foliation, except the leaf at infinity, are \textit{everywhere dense} in $\C^2$ \cite{HudaiVerenov1962,Shcherbakov1984},
 \item There exist a countable number of homologically independent \textit{complex limit cycles} \cite{ShcherbakovRosalesOrtiz1998,GoncharukKudryashov2017}. %SRO prove this only for n>2. Were Yury and Nataliya the first to prove it for n=2?
\end{itemize}

The concept of topological rigidity of polynomial foliations will be discusses in detail in Section \ref{subsec:rigidity} and in Chapter \ref{chpt:holonomy}, where it will be shown that generic quadratic foliations have the strongest possible form of rigidity.




\section{Statement of the results}

There are two results that we present as the main components of this thesis. The first gives necessary and sufficient conditions for two generic quadratic vector fields to be analytically equivalent. This is done in two independent ways: Theorem \ref{thm:moduliSpectra} and Theorem \ref{thm:moduliHolonomy}. Second, we show in Theorem \ref{thm:utmost} that generic quadratic vector fields have the \textit{utmost rigidity property}.

Additionally, we prove a series of results that describe the extended spectra of singularities of generic quadratic vector fields (Theorems \ref{thm:twins}--\ref{thm:noIndexTheorem}).


% \vspace{-\parskip} %yuk!
\subsection{Moduli of analytic classification}

We present two sets of invariants that serve as moduli of analytic classification. It follows from Remark \ref{rmk:scalingFactor} that the passage from affine equivalence to orbital affine equivalence of polynomial vector fields of the same degree is straightforward: two vector fields $v_1$ and $v_2$ are orbitally affine equivalent if and only if there exists $\lambda\in\C\mbox{*}$ such that $v_1$ and $\lambda v_2$ are affine equivalent.

\begin{theorem}[\cite{TwinVectorFields}]\label{thm:moduliSpectra}
 Two generic quadratic vector fields are affine equivalent if and only if their extended spectra of singularities coincide.
\end{theorem}

\begin{theorem}[\cite{UtmostRigidity}]\label{thm:moduliHolonomy}
 Two generic quadratic vector fields are orbitally affine equivalent if and only if their holonomy groups are infinity are analytically conjugate.
\end{theorem}


% \vspace{-\parskip} %yuk!
Genericity in the former theorem is defined by a (complex algebraic) Zariski open dense set in $\V_2$ whose existence is proved but not further described. The genericity assumptions on the latter theorem (as well as those for Theorem \ref{thm:utmost}) are discussed in detail in Chapter \ref{chpt:holonomy}.


\subsection{The phenomenon of topological rigidity}\label{subsec:rigidity}

It was discovered in \cite{Ilyashenko1978} that polynomial foliations exhibit a phenomenon known as \textit{topological rigidity}. The idea of topological rigidity is that topological equivalence of foliations is extremely rare, and almost only happens when the foliations are, in fact, analytically equivalent. The main result in the cited paper is that a generic polynomial foliation is \textit{absolutely rigid}.

\begin{definition}
We say that a foliation $\F\in\A_n$ is \emph{absolutely rigid} if there exist a neighborhood $U$ of $\F$ in $\A_n$ and a neighborhood $V$ of the identity map in the space of self homeomorphisms of $\CP^2$ such that any foliation from $U$ which is conjugate to $\F$ by a homeomorphism in $V$ is necessarily affine equivalent to $\F$.
\end{definition}

The genericity assumptions in \cite{Ilyashenko1978} excluded a dense subset of $\A_n$. These assumptions have been substantially weakened over the years, and it is known now that every foliation in some open and dense subset of $\A_n$ is absolutely rigid ~\cite{Shcherbakov1984,Nakai1994,LinsNetoSadScardua1998}. The key assumption in the latest works is the non-solvability of the holonomy group at infinity. However, the necessity to restrict ourselves to a small neighborhood $U\subset\A_n$ and to assume proximity of the conjugating homeomorphism to the identity map were not dropped.

We prove here that the ideal paradigm of topological rigidity may be formalized for quadratic foliations: topological equivalence implies analytic equivalence --no additional hypotheses are needed.

\begin{theorem}[\cite{UtmostRigidity}]\label{thm:utmost}
 Two generic foliations from the class $\A_2$ are topologically equivalent if and only if they are affine equivalent.
\end{theorem}

Note that non-solvable holonomy groups are topologically rigid, hence topologically equivalent generic foliations have analytically conjugate holonomy groups. For this reason, the above theorem is a direct consequence of Theorem \ref{thm:moduliHolonomy}. However, the question of rigidity was the motivating question that lead to Theorem \ref{thm:moduliHolonomy}.


\subsection{Twin vector fields}

Theorem \ref{thm:moduliSpectra} claims that the \textit{extended} spectra of singularities is a complete set of analytic invariants. A natural question is whether or not the \textit{finite} spectra of singularities (that is, the spectra of the singularities taken only over the affine part) determines completely the vector field (up to affine equivalence). The answer is no --generically, there are two disjoint orbits of the action of the group $\Aff{2}{\C}$ on $\V_2$ consisting of vector fields having the same finite spectra.

\begin{definition}\label{def:twins}
We will say that two vector fields $v_1$ and $v_2$ are \emph{twin vector fields} if they are not equal yet they have exactly the same singular locus and, for each point $p$ in the common singular set, the matrices $Dv_1(p)$ and $Dv_2(p)$ have the same spectrum.
\end{definition}

\begin{theorem}[\cite{TwinVectorFields}]\label{thm:twins}
 A generic quadratic vector field has exactly one twin. Moreover, if two vector fields from the class $\V_2$ have the same finite spectra (no assumption on the position of the singularities) then, after transforming one of them by a suitable affine map, they are either identical or a pair of twin vector fields.
\end{theorem}


\subsection{Relations on spectra and lack of new index theorems}

The extended spectra of a generic quadratic field consists of 11 complex numbers: 8 coming from the finite spectra and 3 characteristic numbers at infinity. These numbers are not independent, they are constrained by four classical \textit{index theorems}: 
\begin{align} 
 \sum_{v(p)=0}\frac{1}{\det Dv(p)} &= 0, \label{eq:firstEJ1}\\
 \sum_{v(p)=0}\frac{\tr Dv(p) }{\det Dv(p)} &= 0, \label{eq:firstEJ2}\\
 \sum_{p\in\operatorname{Sing }\F_v} \hspace{-8pt}\BBindex{\F_v}{p} &= 16, \label{eq:firstBB}\\
 \sum_{p\in \mathcal{L}\cap\operatorname{Sing}\F}\hspace{-12pt}\CSindex{\F_v}{\mathcal{L}}{p} &= 1. \label{eq:CS}
\end{align}
These are, respectively, the Euler-Jacobi equations, the Baum-Bott theorem and the Camacho-Sad theorem. 

Let's do a simple dimension count: on one hand, the space $\V_2$ has dimension 12 and the affine group $\Aff{2}{\C}$ is 6-dimensional. Therefore, the quotient (in the sens of Geometric Invariant Theory) $\V_2\sslash\Aff{2}{\C}$ has dimension 6. On the other hand, the extended spectra, which is of dimension 11, modulo the 4 equations above is a space of dimension 7. This gap in the dimensions implies that there must exist at least one more algebraic relation among these numbers. This \textit{hidden relation} was, until very recently, completely unknown. In order to describe this relation, let us introduce some notation.

Let $v\in\V_2$ have singularities $p_1,\ldots,p_4$ on $\C^2$ and singular points at infinity $q_1,q_2,q_3$. Denote by $(t_k,d_k)$ the spectrum of $v$ at $p_k$, and by $\lambda_j$ the characteristic number of $q_j$. Denote by $\Lambda$ the product $\Lambda=\lambda_1\lambda_2\lambda_3$, and define $S$ to be the graded polynomial ring $S=\C[t_1,\ldots,t_4,d_1,\ldots,d_4]$, where the generators $t_k$ are of degree 1, and $d_k$ of degree 2.

\begin{theorem}[\cite{NoIndexTheorems}]\label{thm:hiddenRelation}
 There exist symmetric polynomials $P_0,P_1,P_2\in S$, homogeneous (with respect to the grading of $S$), such that every generic quadratic vector field satisfies the following equation
  \begin{equation}\label{eq:hiddenRelation}
   P_2(t;d)\,\Lambda^2+P_1(t;d)\,\Lambda+P_0(t;d)=0 .
  \end{equation}
 Moreover, the above relation is independent from the identities (\ref{eq:firstEJ1})--(\ref{eq:CS}).
\end{theorem}

In collaboration with Yuri Kudryashov, we obtained this equation using computer algebra software. However, the polynomial $P_2\Lambda^2+P_1\Lambda+P_0$, which is irreducible, has a very long expression: it consists of approximately four-thousand monomials. The fact that the degree in $\Lambda$ is quadratic is closely related to the fact that generic quadratic vector fields have a unique twin. 

Despite the length of the equation, we have used its explicit expression to show that (\ref{eq:hiddenRelation}) does not come from an index theorem. In fact, we show that any possible ``index-theorem-like identity'' can be deduced from the classical index theorems, hence concluding the lack of existence of new index theorems that constrain the extended spectra of quadratic vector fields.

\begin{theorem}[\cite{NoIndexTheorems}]\label{thm:noIndexTheorem}
 There exists no pair $(R,r)$ consisting of a rational function $R$ on $\C^8$ and a symmetric rational function $r$ on $\C^3$ with the property that every quadratic vector field with non-degenerate singularities satisfies the relation
  \[ R(t;d)=r(\lambda), \]
 except for those that can be derived from the previously known relations: Euler-Jacobi, Baum-Bott and Camacho-Sad.
\end{theorem}










\chapter{The spectra of singularities}

\section{Introduction}

Consider a degree $n$ polynomial vector field on $\C^2$ having only isolated singularities (both in the affine part and on the line at infinity). We have defined in Section \ref{sec:analyticInvariants} the \textit{extended spectra of singularities} to be the collection of the spectra of the linearization matrices of each singular point over the affine part together with all the characteristic numbers at infinity. This collection consists of $2n^2+n+1$ complex numbers, and is invariant under affine equivalence of vector fields.

In this chapter we are going to discuss the four classical index theorems that constrain the extended spectra. After this, we will specialize to the quadratic case and explain up to what extent the spectra determines the vector field (up to affine equivalence). We conclude by describing the last \textit{hidden relation} among the spectra and proving that this equation does not come from an index theorem. In fact, there are no more index theorems constraining the extended spectra of quadratic vector fields, other than the ones described above.


% \section{The classical index theorems}

\subsection{The Euler-Jacobi relations}

Let us recall, in the particular case relevant to us, a classical result known as the Euler-Jacobi formula \cite[Chpt.~5, Sec.~2]{GriffithsHarris1994}.

\begin{theorem}\label{thm:EJformula}
If $P,Q$ are polynomials in $\C[x,y]$ of degree $n$ whose divisors intersect transversely in $n^2$ different points $p_1,\ldots,p_{n^2}\in\C^2$ and $g(x,y)$ is a polynomial of degree at most $2n-3$ then
\[ \sum_{k=1}^{n^2} \frac{g(p_k)}{\mathbf{J}(p_k)}=0, \]
where $\mathbf{J}(x,y)$ is the Jacobian determinant $\mathbf{J}(x,y)=\operatorname{det}\,\displaystyle\frac{\partial(P,Q)}{\partial(x,y)}$.
\end{theorem}

Consider a polynomial vector field $v=P\frac{\partial}{\partial x}+Q\frac{\partial}{\partial y}$ of degree $n\geq2$. By making $g(x,y)=1$ or $g(x,y)=\tr{Dv(x,y)}$ we obtain polynomials whose value at the singular point $p_k$ depends exclusively on the spectrum of $Dv(p_k)$.

\begin{corollary}\label{coro:EJrelations}
A quadratic vector field $v$ having non-degenerate singularities $p_1,\ldots,p_{4}\in\C^2$ satisfies
\begin{align}
 \sum_{k=1}^{4}\frac{1}{\det{Dv(p_k)}}=0, \label{eq:EJ1}\\
 \sum_{k=1}^{4}\frac{\tr{Dv(p_k)}}{\det{Dv(p_k)}}=0. \label{eq:EJ2}
\end{align}
\end{corollary}
\noindent We call these equations the \emph{Euler-Jacobi relations on spectra}.

\begin{remark}
 The \textit{Euler-Jacobi indices} (each summand on the left hand sides of the above identities) should be understood as the \textit{residues} of the rational 2-forms
  \[ \frac{dx\wedge dy}{PQ}, \quad \text{and} \quad \frac{(\tr Dv)\,dx\wedge dy}{PQ} \]
 at the points $p_k$. The \textit{residue theorem} then implies the total sum is zero.
\end{remark}



\subsection{The Baum-Bott theorem}

The Euler-Jacobi indices are well defined for singularities of vector fields, but not for foliations. One of the most important invariants of an isolated singularity of a planar foliation
is the \textit{Baum-Bott index}. Suppose the germ of a foliation $(\F,p)$ with an isolated singularity is given by a holomorphic 1-form $\omega$. The Baum-Bott index of $\F$ at $p$ is defined as
 \[ \BBindex{\F}{p} = \frac{1}{(2\pi i)^2}\int_{\Gamma} \beta\wedge d\beta, \]
where $\Gamma$ is the boundary of a small ball centered at $p$, and $\beta$ is any smooth $(1,0)$-form that satisfies $d\omega=\beta\wedge\omega$ in a neighborhood of $\Gamma$. In the particular case where $\F$ is locally given by a non-degenerate vector field $v$, the index can be easily computed as
 \[ \BBindex{\F}{p} = \frac{\tr^{\,2} Dv(p)}{\det Dv(p)}. \]
 
The Baum-Bott theorem, originally proved in \cite{BaumBott1970} in a more general setting, can be stated in our particular case as follows \cite{Brunella2004b}:

\begin{theorem}
 Let $\F$ be a foliation of projective degree $d$ on $\CP^2$. Then 
  \[ \sum_{p\in\operatorname{Sing}\F} \hspace{-8pt}\BBindex{\F}{p} = (d+2)^2. \]
\end{theorem}

\begin{corollary}
 Let $v\in\V_2$ have finite spectra $\{(t_k,d_k)\}$ and characteristic numbers at infinity $\{\lambda_j\}$. Then
  \begin{equation}
   \sum_{k=1}^4\frac{t_k^2}{d_k} + \sum_{j=1}^3 \frac{(\lambda_j+1)^2}{\lambda_j} = 16 . \label{eq:BB}
  \end{equation}

\end{corollary}

 
\subsection{The Camacho-Sad theorem}

The final classical index theorem, the Camacho-Sad theorem, concerns singularities of a foliation along an invariant curve. Suppose $C$ is a smooth curve invariant by a foliation $\F$ (cf.~\cite{Brunella2004b} for the general case). If $p$ is an isolated singularity of $\F$ on $C$, we can choose a local holomorphic 1-form $\omega$ generating $\F$ and a local equation $f$ for $C$ to obtain a decomposition 
 \[ \omega = h\,df + f\eta, \]
where $h$ is a holomorphic function and $\eta$ a holomorphic 1-form. In this case, the Camacho-Sad index is defined as follows:
 \[ \CSindex{\F}{C}{p} = -\frac{1}{2\pi i}\int_{\gamma} \frac{\eta}{h}, \]
where $\gamma\subset C$ is the boundary of a small disk centered at $p$.

\begin{theorem}[\cite{CamachoSad1982}]
 Let $\F$ be a foliation on a complex surface $S$ and let $C\subset S$ be a compact $\F$-invariant curve. Then
  \[ \sum_{p\in C\cap\operatorname{Sing}\F}\hspace{-12pt}\CSindex{\F}{C}{p} = C\cdot C, \]
 where $C\cdot C$ denotes the self intersection number of $C$ in $S$.
\end{theorem}

\begin{corollary}
  The characteristic numbers at infinity of a vector field $v\in\V_2$ satisfy the relation $\lambda_1+\lambda_2+\lambda_3 = 1$.
\end{corollary}




\section{Twin vector fields}

In this section we prove Theorem \ref{thm:twins}, which claims that a generic quadratic vector field has a unique twin (cf. Definition \ref{def:twins}).

\subsection{The spectra of the fourth singularity}

The first thing we need to observe is the following fact.

\begin{lemma}\label{lemma:3outOf4}
Let $v$ be a quadratic vector field having four non-degenerate singularities $p_1,\ldots,p_4$. The position and spectrum of $p_4$ is completely determined by the position and spectra of $p_1,p_2,p_3$.
\end{lemma}

\begin{proof}
This lemma is a double application of the Euler-Jacobi formula. First, we can think of relations (\ref{eq:EJ1}) and (\ref{eq:EJ2}) as a system of equations which we can solve for $\operatorname{Spec}(v,p_4)$ every time we are given $\operatorname{Spec}(v,p_k)$ for $k=1,2,3$. Second, if we let $g_1(x,y)=x$ and $g_2(x,y)=y$ in the Euler-Jacobi formula in Theorem \ref{thm:EJformula}, then the system of equations 
\[ \sum_{k=1}^4 \frac{g_1(p_k)}{\det{Dv(p_k)}}=0,\qquad \sum_{k=1}^4 \frac{g_2(p_k)}{\det{Dv(p_k)}}=0, \]
determines completely the position of $p_4$.
\end{proof}

Note that the above lemma implies in particular that two quadratic vector fields are twins if and only if three out of their four singularities agree in position and spectra.


\subsection{Proof of the existence and uniqueness of twins}

\begin{proof}[Proof of Theorem \ref*{thm:twins}]
Let us prove the second claim in the theorem first. Suppose that two quadratic vector fields $v,\tilde{v}$ with non-degenerate singularities have the same spectra. After transforming $\tilde{v}$ by a suitable affine map on $\C^2$ we may assume that $v$ and $\tilde{v}$ have three singularities that agree in position and spectra. By Lemma \ref{lemma:3outOf4} the same holds for the fourth singularity. We conclude that either $\tilde{v}=v$ or $\tilde{v}$ is a twin vector field of $v$.

We now prove that twin vector fields exist and are unique. Consider a quadratic vector field $v=P\frac{\partial}{\partial x}+Q\frac{\partial}{\partial y}$ having four non-degenerate singularities $p_1,\ldots,p_4$. By Max Noether's theorem, any quadratic polynomial $H$ which vanishes on the singular set $\operatorname{Sing}\,v=\{P=0\}\cap\{Q=0\}$ can be written uniquely as
\[ H=\alpha P+\beta Q, \]
for some complex numbers $\alpha,\beta$. This means that any quadratic vector field that vanishes on the singular set $\operatorname{Sing}\,v$ can be uniquely written as
\[ \tilde{v}=(aP+bQ)\,\frac{\partial}{\partial x}+(cP+dQ)\,\frac{\partial}{\partial y}, \]
for complex numbers $a,b,c,d$. Let us denote by $A$ the matrix $A=\left(\begin{smallmatrix}a&b\\c&d\end{smallmatrix}\right)$ and note that $D\tilde{v}(x,y)=A\cdot Dv(x,y)$. In virtue of Lemma \ref{lemma:3outOf4} the vector field $\tilde{v}$ has the same spectra as $v$ if and only if they have the same spectra at $p_1,p_2,p_3$. This happens if and only if
\begin{equation}\label{eq:system} \begin{array}{rcl}
\tr{A\cdot Dv(p_k)}&=&\tr{Dv(p_k)},\quad \mbox{for } k=1,2,3, \\
\det{A} &=& 1. \\
\end{array} \end{equation}
The above is a system of three linear equations and one quadratic equation on $a,b,c,d$. If the system is independent, we can always eliminate three of these variables using the linear equations and then substitute into the quadratic one. This gives a quadratic equation in one variable which generically would have two different solutions. 

In order to check that for a generic quadratic vector field the linear system is independent and the discriminant of the quadratic equation is not zero it is enough to find a single example with these properties. A simple example is given by the Hamiltonian vector field 
\begin{equation}\label{eq:Hamiltonianvf}
v=x(x+2y-1)\frac{\partial}{\partial x}+y(-2x-y+1)\frac{\partial}{\partial y}.
\end{equation}
 The computations are straightforward and we will omit them here.

This proves that for a generic vector field system (\ref{eq:system}) has two solutions; one corresponds to the original vector field $v$ and the other to a different vector field $\tilde{v}$, thus establishing existence and uniqueness of a twin vector field for a generic vector field $v$.
\end{proof}




\section{The spectra as moduli of analytic classification}

The aim of this section is to prove Theorem \ref{thm:moduliSpectra}. The spirit of the strategy is straightforward: define a \textit{moduli map} from the space $\V_2$ to the space of possible values for the extended spectra and show that the generic fiber of this map consists of a single point. Since affine equivalent vector fields have the same spectra, we make use of the affine group to normalize the vector fields beforehand.


\subsection{Explicit expressions for the spectra}

\begin{remark}
It is straightforward to show that if a polynomial vector field on $\C^2$ of degree $n$ has $n+1$ collinear singularities then the whole line through these points is singular itself. In particular, a vector field from the class $\V_2$ cannot have three collinear singularities, and hence we have the following lemma.
\end{remark}


\begin{lemma}\label{lemma:normalizeSingularities}
 Every quadratic vector field with four isolated singularities is affine equivalent to a vector field with singularities at $p_1=(0,0)$, $p_2=(1,0)$, $p_3=(0,1)$. Any such vector field $v=P\frac{\partial}{\partial x}+Q\frac{\partial}{\partial y}$ is defined by polynomials
\begin{equation}\label{eq:coefficients}
\begin{array}{l} 
 P(x,y) = a_0x^2+a_1xy+a_2y^2-a_0x-a_2y,  \\ 
 Q(x,y) = a_3x^2+a_4xy+a_5y^2-a_3x-a_5y, \\
\end{array}
\end{equation}
for some complex numbers $a_0,\ldots,a_5$.
\end{lemma}

Under this ``normal form'' we can immediately compute the explicit expressions for the traces $t_k$ and determinants $d_k$ at each singular point $p_1,p_2,p_3$.

\begin{lemma}\label{lemma:expressionsSpectra}
 The spectra $(t_k,d_k)$ of a vector field $v$ as in Lemma \ref{lemma:normalizeSingularities} at the singular point $p_k$ is given as follows.
 \begin{align*}
  t_1 &= -a_0-a_5  	& d_1 &= -a_2a_3+a_0a_5 \\
  t_2 &= a_0+a_4-a_5  	& d_2 &= -a_1a_3+a_2a_3+a_0a_4-a_0a_5 \\
  t_3 &= -a_0+a_1+a_5  	& d_3 &= a_2a_3-a_2a_4-a_0a_5+a_1a_5 
 \end{align*}
\end{lemma}

\begin{lemma}\label{lemma:expressionLambda}
 Let $\Lambda=\lambda_1\lambda_2\lambda_3$ denote the product of all characteristic numbers at infinity of some vector field $v$ as in Lemma \ref{lemma:normalizeSingularities}. In the generic case, $\Lambda$ can be computed as follows:
 \begin{equation}\label{eq:expressionLambda}
  \Lambda = -27\,\frac{d_1d_2+d_2d_3+d_3d_1}{g(a)},
 \end{equation}
 where $g(a)$ is given by
 \begin{align*}
  g(a) &= 4a_0^3a_2-a_0^2a_1^2+2a_0^2a_1a_5-12a_0^2a_2a_4-a_0^2a_5^2+2a_0a_1^2a_4+18a_0a_1a_2a_3 \\
  & -4a_0a_1a_4a_5-18a_0a_2a_3a_5+12a_0a_2a_4^2+2a_0a_4a_5^2-4a_1^3a_3+12a_1^2a_3a_5 \\
  & -a_1^2a_4^2-18a_1a_2a_3a_4-12a_1a_3a_5^2+2a_1a_4^2a_5+27a_2^2a_3^2+18a_2a_3a_4a_5 \\
  & -4a_2a_4^3+4a_3a_5^3-a_4^2a_5^2.
 \end{align*}
\end{lemma}

\begin{proof}
 In coordinates $z=\displaystyle\frac{1}{x}$, $w=\displaystyle\frac{y}{x}$ the foliation is given by 
  \[ \frac{dz}{dw} = z\,\frac{s(w)}{r(w)} + O(z^2), \]
 where the roots $w_k$ of $r(w)$ are the singular points at infinity, and  $\lambda_k=\displaystyle\frac{s(w_k)}{r'(w_k)}$. We obtain the formula
  \[ \Lambda = \frac{s(w_1)s(w_2)s(w_3)}{r'(w_1)r'(w_2)r'(w_3)}. \]
  
 The numerator in the above expression is the product of $s(w)$ at each of the roots of $r$, hence is given by the resultant $\Res(s,r)$.  This is true since $r$ and $s$ are both monic polynomials. Taking into account that $r'$ has leading coefficient equal to $3$ and that $r$ is cubic, we obtain that the denominator equals $\frac{1}{27}\Res(r',r)$. A simple computation now shows that 
  \[ \Res(s,r)=-a_2\cdot f(a), \qquad \Res(r',r)=a_2\cdot g(a), \]
 where $f(a)$ is a polynomial in $a$ that can be verified to satisfy $f(a)=d_1d_2+d_2d_3+d_3d_1$ (where the expressions for $d_k$ are those in Lemma \ref{lemma:expressionsSpectra}) and $g(a)$ is given in the statement of the current lemma. We conclude that $\Lambda=-27\frac{f(a)}{g(a)}$ as claimed.
\end{proof}

\begin{remark}\label{rmk:determineLambdas}
 Note that if $v$ is a vector field as in Lemma \ref{lemma:normalizeSingularities} and we know the value of $g(a)$ and $\{(t_k,d_k)\}$, we can always recover the value of $\Lambda$. Moreover, using the Baum-Bott equation (\ref{eq:BB}) and the Camacho-Sad relation, we can recover the exact value of $\lambda_1,\lambda_2,\lambda_3$ from $\{(t_k,d_k)\}$ and $\Lambda$.
\end{remark}


\subsection{The moduli map} 

\begin{definition}\label{def:moduliMap}
We will call \textit{moduli map} regular map $\mathcal{M}\colon\C^6\to\C^7$ that takes 
\[ (a_0,\ldots,a_5)\longmapsto (t_1,\ldots,t_3,d_1,\ldots,d_3,g(a)).  \]
The explicit expression of $g(a)$ is given in Lemma \ref{lemma:expressionLambda}.
\end{definition}

Note that in virtue of Lemma \ref{lemma:3outOf4} and Remark \ref{rmk:determineLambdas}, the value of $\mathcal{M}(a)$ is enough to determine the precise value of the extended spectra of the vector field $v$ defined by the coefficients $a_0,\ldots,a_5$.

In order to prove Theorem \ref{thm:moduliSpectra} we need to show that the generic fiber of the moduli map $\mathcal{M}$ consists of a single point. In order to do this it is enough to find a non-empty open set $W\subset\operatorname{Im }\mathcal{M}$ and $U\subset\C^6$ such that $U=\mathcal{M}^{-1}(W)$ and $\mathcal{M}\vert_{U}\colon U\to W$ is one-to-one.

\begin{proof}[Proof of Theorem \ref*{thm:moduliSpectra}]
Consider the following vector field
\[ v_0 = (x^2+2xy-x)\frac{\partial}{\partial x}+(-xy+3y^2-3y)\frac{\partial}{\partial y}. \]
Let us denote by $\operatorname{Spec}$ the function $\operatorname{Spec}(a)=(t_1,\ldots,d_3)$, and let $S_0=\operatorname{Spec}(v_0)$. A simple computation shows that the unique twin vector field of $v_0$ is
\[ v_1 = (3x^2+6xy-3x)\frac{\partial}{\partial x}+\left(-\frac{7}{3}x^2-5xy+y^2+\frac{7}{3}x-y\right)\frac{\partial}{\partial y}, \]
and that the derivative of the map $\operatorname{Spec}$ is invertible both at $v_0$ and at $v_1$. We can deduce from the inverse function theorem the existence of neighborhoods $\widetilde{W}, U_0, U_1$ of $S_0,v_0,v_1$ respectively such that $\operatorname{Spec}^{-1}(\widetilde{W})= U_0\cup U_1$ and $\operatorname{Spec}$ maps both $ U_0$ and $ U_1$ diffeomorphically onto $\widetilde{W}$. On the other hand, it is not hard to check that the product of the characteristic numbers at infinity of $v_0$ and $v_1$ are different. By shrinking $ U_0$ and $ U_1$ if necessary we can assume that vector fields in $ U_0$ have different product of characteristic numbers from any vector field in $ U_1$. This means that if $\Lambda_0$ is the product of characteristic numbers of $v_0$, we can find a small neighborhood $V_0$ of $\Lambda_0$ such that no vector field from $ U_1$ has product of characteristic numbers in $V_0$. Define $W=\widetilde{W}\times V_0\subset\C^6\times\C$. Since $\mathcal{M}=(\operatorname{Spec},\Lambda)$ and $\operatorname{Spec}^{-1}(\widetilde{W})= U_0\cup U_1$, we must have $\mathcal{M}^{-1}(W)\subset  U_0\cup U_1$. However we also know that $\mathcal{M}^{-1}(W)$ is disjoint from $ U_1$, by construction of $V_0$, and so $\mathcal{M}^{-1}(W)\subset U_0$. If we let $U=\mathcal{M}^{-1}(W)$ we have that $\mathcal{M}\vert_{U}\colon U\to W$ is one-to-one, and so the generic fiber of $\mathcal{M}$ consists of a single point.
\end{proof}


\subsection{A remark about the Baum-Bott indices}

We want to point out that Theorem \ref{thm:moduliSpectra} is very similar in spirit to results in \cite{LinsNeto2012} (for foliations on $\CP^2$ of degree two) and \cite{IlyashenkoMoldavskis2011} (for foliations on $\CP^2$ coming from a generic quadratic vector field on $\C^2$), where it is proved that in the generic case the Baum-Bott indices completely determine a foliation \textit{up to finite ambiguity} (modulo the natural action of $\PGL{2}{\C}$ and $\Aff{2}{\C}$, respectively). In fact, Lins Neto proves that the generic fiber of the Baum-Bott map, which associates to a foliation the Baum-Bott indices of its singularities, contains exactly 240 orbits of the natural action of $\PGL{2}{\C}$.



\section{The hidden relation and lack of new index theorems}

Theorem \ref{thm:moduliHolonomy} has the consequence that the space of possible extended spectra of generic quadratic vector fields is 6-dimensional. This is of course the expected dimension, since the quotient $V_2\sslash\Aff{2}{\C}$ is 6-dimensional itself. The extended spectra consists of 11 numbers: 8 coming from the finite spectra $\{(t_k,d_k)\}$, and 3 coming from the singular points at infinity $\{\lambda_j\}$. These 11 numbers are related by 4 classical index theorems: the Euler-Jacobi relations, the Baum-Bott equation and the Camacho-Sad formula. However, there must exist one more algebraic relation, independent of the previous ones, that constrains these numbers.

It was proved in \cite{WoodsHole} that all four classical relations can be realized as particular cases of the so-called \textit{Woods Hole trace formula}, also known as the \textit{Atiyah-Bott fixed point theorem}, and it was expected that the missing hidden relation would also be of this type. This is not the case. As it will be shown in this section, there are no more index theorems that relate the extended spectra other than those which can derived from the classical ones.


\subsection{Description of the hidden relation}

The moduli map is a regular map between affine spaces. This map induces a ring map on the corresponding coordinate rings. These rings are $R=\C[a_0,\ldots,a_5]$ and $S=\C[t_1,\ldots,t_3,d_1,\ldots,d_3,g]$ respectively. We grade $S$ by declaring that the $t_k$ are of degree 1, $d_k$ of degree 2, and $g$ of degree 4. 

\begin{remark}
 The regular map $\mathcal{M}\colon\mathbb{A}^6\to\mathbb{A}^7$ induces a ring map $\mathcal{M}^*\colon S\to R$. The closure of the image of $\mathcal{M}$ is the subvariety of $V\subset\mathbb{A}^7$ defined by the ideal $I=\operatorname{Ker}\,\mathcal{M}^*$. Moreover, since $V$ is a hypersurface on $\mathbb{A}^7$, $I$ is the principal ideal generated by some polynomial $F\in \C[t_1,\ldots,t_3,d_1,\ldots,d_3,g]$.
\end{remark}

In order to find the polynomial $F$, we need only ask a computer algebra software to compute a basis for the ideal $I$. This can easily be done by any software that handles Gr\"{o}bner bases. We have done this using both \textit{Macaulay2} and \textit{CoCoA}, the resulting generator is the same. The explicit expression of this polynomial may be found in the appendix.

\begin{proposition}
 The ideal $I=\operatorname{Ker}\,\mathcal{M}^*$ is generated by a single polynomial $F=A(t;d)\,g^2+B(t;d)\,g+C(t;d)$. This polynomial has the following properties:
 \begin{itemize}[topsep=-2pt,itemsep=-4pt]
  \item $F$ is irreducible in $S$,
  \item $F$ is homogeneous of degree 14 (with respect to the grading of $S$),
  \item $A=8(d_1+d_2)(d_2+d_3)(d_3+d_1)$,
  \item $B$ and $C$ are irreducible in $S$.
 \end{itemize}
\end{proposition}

By the definition of $g$ in Lemma \ref{lemma:expressionLambda}, in order to obtain the hidden relation we need only substitute
 \[ g=-27\frac{d_1d_2+d_2d_3+d_3d_1}{\Lambda} \]
in $A(t;d)\,g^2+B(t;d)\,g+C(t;d)=0$ and multiply by $\Lambda^2$ to lift denominators. Explicitly, we have the following:
\begin{align*} 
 P_0 &= 729(d_1d_2+d_2d_3+d_3d_1)^2A , \\
 P_1 &= -27(d_1d_2+d_2d_3+d_3d_1)B, \\
 P_2 &= C .
\end{align*}
This gives the following equation with all the properties listed in Theorem \ref{thm:hiddenRelation}
 \[ P_2(t;d)\,\Lambda^2+P_1(t;d)\,\Lambda+P_0(t;d)=0. \]
% Should I explain the independence from the other theorems in more detail???
 

\subsection{Lack of new index theorems}

foo





\chapter{The holonomy group at infinity}\label{chpt:holonomy}
% 
% \appendix
% \chapter{Mathematca script?}


\nocite{StrongTopoInvariance,UtmostRigidity,TwinVectorFields,WoodsHole,NoIndexTheorems}
\bibliography{ref-valente}

\end{document}







%%COMMENTS%%

% Do I want to switch CP2 by P2?
% I introduced the definition ``Spec p'' as Spec Dv(p). Make sure it is used in Chapter 2.
% Excesive amount of space in display math mode can be fixed with \singlespacing.
% Check the use of ``extended spectra''.
% I'm using \lambda_j for the residues but Twins uses \lambda_j.
% Careful: we don't want \qed commands
% Check the degrees of the P_k in hidden relation.
% Add explicit expression of F in the appendix.










