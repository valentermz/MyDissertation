\lstset{language=Mathematica,mathescape}
\lstset{basicstyle={\sffamily\footnotesize},
%   numbers=left,
%   numberstyle=\tiny\color{gray},
  numbersep=5pt,
  breaklines=true,
  captionpos={t},
%   frame={lines},
%   rulecolor=\color{black},
%   framerule=0.5pt,
  columns=flexible,
  tabsize=2
}

\section{Existence and uniqueness}

The following is a \textit{Mathematica} script containing explicitly the computations performed in Section \ref{subsec:twins} to prove the existence and uniqueness of twin vector fields.

\bigskip
\bigskip

\begin{lstlisting}
<< Notation`
Symbolize[ParsedBoxWrapper[SubscriptBox["a", "_"]]]

p$_1$={0,0};
p$_2$={1,0};
p$_3$={0,1};

P[{x_,y_}]=a$_0$*x^2+a$_1$*x*y+a$_2$*y^2-a$_0$*x-a$_2$*y;
Q[{x_,y_}]=a$_3$*x^2+a$_4$*x*y+a$_5$*y^2-a$_3$*x-a$_5$*y;

Dv[{x_,y_}]={{D[P[{x,y}],x],D[P[{x,y}],y]},{D[Q[{x,y}],x],D[Q[{x,y}],y]}};

t$_1$=Tr[Dv[p$_1$]];
d$_1$=Det[Dv[p$_1$]];
t$_2$=Tr[Dv[p$_2$]];
d$_2$=Det[Dv[p$_2$]];
t$_3$=Tr[Dv[p$_3$]];
d$_3$=Det[Dv[p$_3$]];

M = {{$\alpha$, $\beta$}, {$\gamma$, $\delta$}};

eq$_1$ = Tr[M.Dv[p$_1$]] - t$_1$;
eq$_2$ = Tr[M.Dv[p$_2$]] - t$_2$;
eq$_3$ = Tr[M.Dv[p$_3$]] - t$_3$;
\end{lstlisting}

\noindent\makebox[\linewidth]{\rule{\textwidth}{0.4pt}}
\begin{lstlisting}
Solve[eq$_1$ == 0 && eq$_2$ == 0 && eq$_3$ == 0, {$\beta$, $\gamma$, $\delta$}]
values = %[[1]]
\end{lstlisting}
\begin{small}
 \begin{align*}
  \big\{\big\{ \beta &\to -\frac{-2a_0+a_1+2a_0\alpha-a_1\alpha}{2a_3-a_4} \\
  \gamma &\to - \big(-a_1 a_3 a_4+a_0 a_4^2+2 a_1 a_3 a_5-2 a_0 a_4 a_5+a_1 a_3 a_4 \alpha-a_0 a_4^2 \alpha-2 a_1 a_3 a_5 \alpha \\
  &\phantom{\gamma\to} +2 a_0 a_4 a_5 \alpha\big) / \big((2a_3-a_4)(a_2a_4-a_1a_5)\big) \\
  \delta &\to \big(-a_1^2 a_3+2 a_1 a_2 a_3+a_0 a_1 a_4-2 a_0 a_2 a_4+2 a_2 a_3 a_4-a_2 a_4^2-2 a_1 a_3 a_5+a_1 a_4 a_5 \\
  &\phantom{\delta\to} +a_1^2 a_3 \alpha-2 a_1 a_2 a_3 \alpha-a_0 a_1 a_4 \alpha+2 a_0 a_2 a_4 \alpha\big) / \big((2a_3-a_4)(a_2a_4-a_1a_5)\big) \big\}\big\}
 \end{align*}
\end{small}
\noindent\makebox[\linewidth]{\rule{\textwidth}{0.4pt}}

\begin{lstlisting}
Solve[detM == 1, {$\alpha$}]
\end{lstlisting}
\begin{small}
 \begin{align*}
  \big\{\{ \alpha &\to 1 \} \\
  \{\alpha &\to \big(-2 a_0 a_1 a_3 a_4+a_1^2 a_3 a_4 -4 a_2 a_3^2 a_4+2 a_0^2 a_4^2-a_0 a_1 a_4^2+4 a_2 a_3 a_4^2-a_2 a_4^3 \\
  & +4 a_0 a_1 a_3 a_5-2 a_1^2 a_3 a_5+4 a_1 a_3^2 a_5-4 a_0^2 a_4 a_5+2 a_0 a_1 a_4 a_5 -4 a_1 a_3 a_4 a_5 \\
  &+a_1 a_4^2 a_5\big) / \big(2 (a_1 a_3-a_0 a_4) (a_1 a_3-2 a_2 a_3-a_0 a_4+a_2 a_4+2 a_0 a_5-a_1 a_5)\big) \}\big\} \phantom{\alpha}
 \end{align*}
\end{small}
\noindent\makebox[\linewidth]{\rule{\textwidth}{0.4pt}}

\section{The discriminant}

The expression found for $\alpha$ above is a rational function on $a_0,\ldots,a_5$, even though we knew that the equation $\det M = 1$ is quadratic. Below we verify that the discriminant $\Delta$ of the equation $\det M = 1$ factors as a perfect square and so the numbers $\alpha,\ldots,\delta$ belong indeed to the field of fractions of the coefficients $a_0,\ldots,a_5$.

\bigskip
\bigskip

\begin{lstlisting}
a=SeriesCoefficient[detM-1,{$\alpha$,0,2}];
b=SeriesCoefficient[detM-1,{$\alpha$,0,1}];
c=SeriesCoefficient[detM-1,{$\alpha$,0,0}];
\end{lstlisting}

\noindent\makebox[\linewidth]{\rule{\textwidth}{0.4pt}}
\begin{lstlisting}
$\Delta$=Factor[b^2-4*a*c]
\end{lstlisting}
\begin{small}
 \begin{equation*}
  \Delta = \frac{\left(a_1^2 a_3 +a_2 (2 a_0+2 a_3-a_4) a_4-a_1 (2 a_2 a_3+a_0 a_4+2 a_3 a_5-a_4 a_5)\right)^2}{(2a_3-a_4)^2 (a_2a_4-a_1a_5)^2}
 \end{equation*}
\end{small}
\noindent\makebox[\linewidth]{\rule{\textwidth}{0.4pt}}























